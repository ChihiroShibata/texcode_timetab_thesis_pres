\documentclass{jarticle}[10pt]
\usepackage[utf8]{inputenc}
\usepackage{tabularx, multirow, multicol}
\usepackage{booktabs}
\usepackage[margin=1.5cm]{geometry}

\title{卒業論文中間発表会 (2023)\\タイムテーブル}
\author{金沢研究室 / 柴田研究室}
\date{実施日 : 2023年12月21日(木)}

\begin{document}

\maketitle
\begin{itemize}
\item 場所は C203 (中央館1F学生演習室) です. 
\item 持ち時間は,一人あたり,発表 9 分 + 質問 3 分で,合計 12 分です.
\end{itemize}
\begin{table}[h]
\centering
% タイムテーブル
\begin{tabularx}{\textwidth}{l l X l}
\toprule
\multicolumn{1}{c}{\textbf{時間}} &
\multicolumn{1}{c}{\textbf{発表者}} &
\multicolumn{1}{c}{\textbf{題目}} &
\multicolumn{1}{c}{\textbf{研究室}}  \\
\toprule
13:20 - 13:32 & 池田 茉央 &  & \multirow{3}{*}[-0.45em]{金沢 研究室} \\
\cmidrule(r){1-3}
13:32 - 13:44 & 牧野 駿人 &  & \\
\cmidrule(r){1-3}
13:44 - 13:56 & 山田 真生 &  & \\
\toprule
13:56 - 14:08 & 川口 智輝 & M1グランプリをもとにした面白い漫才の自動生成 & \multirow{4}{*}[-0.7em]{柴田 研究室} \\
\cmidrule(r){1-3}
14:08 - 14:20 & 高石 純平 & 日本語LLMを用いた俳句の自動生成 & \\
\cmidrule(r){1-3}
14:20 - 14:32 & 川島 尚大 & 深層学習モデルを用いた子宮鏡画像による慢性子宮内膜炎の分類 & \\
\cmidrule(r){1-3}
14:32 - 14:44 & 寺澤 佳祐 & 慢性子宮内膜炎の異常検知 & \\
\toprule
14:44 - 14:56 & \multicolumn{3}{c}{休憩} \\
\toprule
14:56 - 15:08 & 風巻 武尊 &  & \multirow{3}{*}[-0.45em]{金沢 研究室} \\
\cmidrule(r){1-3}
15:08 - 15:20 & 三溝 寛大 &  & \\
\cmidrule(r){1-3}
15:20 - 15:32 & 城戸 道仁 &  & \\
\toprule
15:32 - 15:44 & 村瀬 響基 & 敵対的生成ネットワークを用いた胎児心拍数陣痛図のノイズ除去 & \multirow{4}{*}[-0.7em]{柴田 研究室} \\
\cmidrule(r){1-3}
15:44 - 15:56 & 乃坂 京介 & 特定の作者の画風を学習・画像に反映させる研究 & \\
\cmidrule(r){1-3}
15:56 - 16:08 & 堀川 剛  & ゆるキャラの画像生成 & \\
\cmidrule(r){1-3}
16:08 - 16:20 & 川原 多聞 & 映像予測タスクにおける生成映像の高精細化 & \\
\bottomrule
\end{tabularx}
\end{table}

\end{document}