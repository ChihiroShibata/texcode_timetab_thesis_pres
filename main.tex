\documentclass{jarticle}[10pt]
\usepackage[utf8]{inputenc}
\usepackage{tabularx, multirow, multicol}
\usepackage{booktabs}
\usepackage[margin=1.5cm]{geometry}

\title{卒業論文中間発表会 (2023)\\タイムテーブル}
\author{金沢研究室 / 柴田研究室}
\date{実施日: 2023年12月21日 (木)}

\begin{document}

\maketitle
\begin{itemize}
\item 場所は C203 (中央館1F学生演習室) です. 
\item 持ち時間は,一人あたり,発表 9 分 + 質問 3 分で,合計 12 分です.
\item 発表者の方は発表会開始前ないし休憩時間に,接続チェックを行ってください.
\end{itemize}
\begin{table}[h]
\centering
% タイムテーブル
\begin{tabularx}{\textwidth}{l l X l}
  \toprule
  \multicolumn{1}{c}{\textbf{時間}} &
  \multicolumn{1}{c}{\textbf{発表者}} &
  \multicolumn{1}{c}{\textbf{題目}} &
  \multicolumn{1}{c}{\textbf{研究室}}  \\
  \toprule
  13:30 -- 13:42 & 池田 茉央 & 小町算を解くプログラムの証明について & \multirow{3}{*}[-0.45em]{金沢 研究室} \\ \cmidrule(r){1-3}
  13:42 -- 13:54 & 牧野 駿人 & プログラミング言語Agdaでカウントダウン問題を検証する & \\ \cmidrule(r){1-3}
  13:54 -- 14:06 & 山田 真生 & 正規形の性質を用いたカウントダウン問題の解法のAgdaによる検証 & \\ \midrule
  14:06 -- 14:12 & \multicolumn{3}{c}{交代(6分)} \\ \midrule
  14:12 -- 14:24 & 川口 智輝 & M1グランプリをもとにした面白い漫才の自動生成 & \multirow{4}{*}[-0.7em]{柴田 研究室} \\ \cmidrule(r){1-3}
  14:24 -- 14:36 & 高石 純平 & 日本語LLMを用いた俳句の自動生成 & \\ \cmidrule(r){1-3}
  14:36 -- 14:48 & 川島 尚大 & 深層学習モデルを用いた子宮鏡画像による慢性子宮内膜炎の分類 & \\ \cmidrule(r){1-3}
  14:48 -- 15:00 & 寺澤 佳祐 & 慢性子宮内膜炎の異常検知 & \\ \toprule
  15:00 -- 15:12 & \multicolumn{3}{c}{休憩(12分)} \\ \toprule
  15:12 -- 15:24 & 風巻 武尊 & 小町算プログラムの正当性の証明 & \multirow{4}{*}[-0.45em]{金沢 研究室} \\ \cmidrule(r){1-3}
  15:24 -- 15:36 & 三溝 寛大 & 最大非連続部分列和問題の線形時間アルゴリズムの証明 & \\ \cmidrule(r){1-3}
  15:36 -- 15:48 & 城戸 道仁 & 重複を除去しつつ辞書順で最小のものを返す効率的なアルゴリズムのAgdaによる証明 & \\ \midrule
  15:48 -- 15:54 & \multicolumn{3}{c}{交代(6分)} \\ \midrule
  15:54 -- 16:06 & 村瀬 響基 & 敵対的生成ネットワークを用いた胎児心拍数陣痛図のノイズ除去 & \multirow{4}{*}[-0.7em]{柴田 研究室} \\ \cmidrule(r){1-3}
  16:06 -- 16:18 & 乃坂 京介 & 特定の作者の画風を学習・画像に反映させる研究 & \\ \cmidrule(r){1-3}
  16:18 -- 16:30 & 堀川 剛 & ゆるキャラの画像生成 & \\ \cmidrule(r){1-3}
  16:30 -- 16:42 & 川原 多聞 & 映像予測タスクにおける生成映像の高精細化 & \\
  \bottomrule
\end{tabularx}

\end{table}

\end{document}